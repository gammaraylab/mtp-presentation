% jni_static_analysis.tex
\begin{frame}
    \centering
    \textbf\ttfamily{\huge{Java Native Interface}}
\end{frame}

\begin{frame}{Java Native Interface}
    \begin{itemize}
        \vspace{3mm}
        \item The Java Native Interface (JNI) acts as a bridge between Java and native languages.
        \vspace{3mm}
        \item Enables Java programs to leverage the advantages of native code.
        \vspace{3mm}
        \item Similar to polyglot systems but does not have a shared runtime.
    \end{itemize}
\end{frame}


\begin{frame}[fragile]{Example of JNI in Java}
    \begin{center}
        \begin{minipage}{0.8\textwidth} % Adjust width as needed
            \begin{lstlisting}[language=java,
                frame=single,
                numbers=left,
                basicstyle=\tiny\ttfamily,
                caption={Vector arithmetic  },
                captionpos=b,
                backgroundcolor=\color{lightgray},   % Background color
                numberstyle=\color{pink},            % Line number color
                keywordstyle=\color{keywordcolor}\bfseries, % Keywords color
                commentstyle=\color{commentcolor},   % Comment color
                stringstyle=\color{stringcolor}      % String color
            ,label={lst:lstlisting}]
public class Main {
    static { // Load the native library
        System.loadLibrary("Polynative");
    }
    // Declare the native method for addition
    public native Vectors add(Vectors a,Vectors b);

    public static void main(String[] args) {
        Vectors p1 = new Vectors(31, 2);    //O9
        Vectors p2 = new Vectors(4, 54);    //O10

        Main caller=new Main();             //O12
        Vectors result1 = caller.add(p1,p2);  // Add p1 and p2
        result1.display();

        for (int i = 0; i < 10000; i++) {
            Vectors temp = new Vectors(i*i, 3*i); //O17
            Vectors sum = caller.add(p1,temp);
            sum.display();
        }
    }
}
            \end{lstlisting}
        \end{minipage}
    \end{center}
%    This example shows a Java HashMap being passed to a Python function, which retrieves and prints values.
\end{frame}

\begin{frame}{Escape Analysis in Java}
    Object is said to be escaping from its allocating method if and only if:
    \begin{itemize}
        \vspace{3mm}
        \item Object is assigned to a static(global) var.
        \vspace{3mm}
        \item It is accessible from another thread.
        \vspace{3mm}
        \item It is reachable via some chain of field dereferences.
        \vspace{3mm}
        \item Passed to a method as parameter which cannot be analyzed.
    \end{itemize}
\end{frame}

\begin{frame}{Escape Analysis in JNI Programs}
    We modified an existing$_2$ inter-procedural escape analysis to collect the following information:
    \begin{itemize}
        \vspace{3mm}
        \item Recording different types of calls(Static, Virtual, Special and Native).
        \vspace{3mm}
        \item Identifying objects escaped via native calls.
    \end{itemize}
     % Smaller font size for the source
    \vspace{3cm}
    \tiny
    \textit{2: \url{https://github.com/meetesh06/}}
\end{frame}

\begin{frame}{Findings using SOOT Analysis Framework}
    \begin{table}[H]
        \label{tab:jni_analysis_results}
        \begin{tabular}{|l|c|c|c|c|c|c|}
            \hline
            \textbf{Parameters} & \textbf{avrora} & \textbf{h2} & \textbf{batik}  & \textbf{helloWorld} & \textbf{myJNI} \\
            \hline
            Classes loaded          & 674  & 753  & 927  & 809  & 811  \\
            Interface calls          & 1499  & 2159  & 1360  & 6586  & 6857  \\
            Special calls           & 1745  & 2062  & 2840   & 2314  & 2329  \\
            JNI calls               & 42   & 53   & 54    & 65   & 65   \\
            Virtual calls           & 11206 & 13027 & 14293  & 13358 & 14826 \\
            Static calls            & 1124 & 1288  & 1715  & 1283  & 1286  \\
            Total objects          & 934  & 1122  & 1068  & 1210  & 1223 \\
            Escaped objects         & 531  & 459  & 768    & 569  & 574  \\
            Escaped via JNI         & 180   & 82   & 302     & 151   & 151   \\
            \% Escaped via JNI     & 33\%  & 17\%  & 39\%   & 26\%  & 26\%   \\
            Methods analyzed        & 2987 & 3539 & 4310 & 4033 & 4051 \\
            \hline
        \end{tabular}
        \caption{Results for some popular benchmark from $Dacapo$ \& custom programs}
    \end{table}

\end{frame}

\begin{frame}{Summary}
       \begin{itemize}
        \vspace{3mm}
        \item Experiment revealed that JNI calls are relatively rare.
        \vspace{3mm}
        \item The number of escaped objects through JNI calls were significant($~28\%$).
        \vspace{3mm}
        \item A $Hello world!$ program implicitly invokes JNI methods via java libraries.
        \vspace{3mm}
        \item These calls make results imprecise, since analysis is limited by language boundary.
        \vspace{3mm}
        \item If we could incorporate native methods analysis, optimization techniques could be improved.
    \end{itemize}
\end{frame}
