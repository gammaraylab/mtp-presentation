% 1_introduction.tex

\begin{frame}{What are Polyglot Programs}
    \begin{itemize}
        \item Programs having multiple languages within a single program.
        \vspace{3mm}
        \item Programs that run in a shared runtime.
        \vspace{3mm}
    \end{itemize}
\end{frame}

\begin{frame}[fragile]{Example 1: Calling a Python function in Java}
    \begin{center}
        \begin{minipage}{0.8\textwidth} % Adjust width as needed
        \begin{lstlisting}[language=java,
            frame=single,
            label={lst:pythonInJava},
            numbers=left,
            basicstyle=\tiny\ttfamily,
            caption={Passing java's HashMap object to python},
            captionpos=b,
            backgroundcolor=\color{lightgray},   % Background color
            numberstyle=\color{pink},            % Line number color
            keywordstyle=\color{keywordcolor}\bfseries, % Keywords color
            commentstyle=\color{commentcolor},   % Comment color
            stringstyle=\color{stringcolor}      % String color
        ]
    public static void main(String[] args) {
        HashMap<String, Object> input = new HashMap<>();
        input.put("college", "IIT Bombay");
        input.put("location", "Mumbai");
        javaMapInPython(input);
    }

    @FunctionalInterface interface MapFunction {
        Map printObjects(HashMap input);
    }

    public static void javaMapInPython(HashMap<String,Object> javaMap) {
        Context context = Context.newBuilder().allowHostAccess(true).build();
        Value function = context.eval("python",
        "def printObjects(map):\n" +
        "   print(map.get('college'))\n" +
        "   print(map.get('location'))\n");

        function.as(MapFunction.class).printObjects(javaMap);
    }
        \end{lstlisting}
        \end{minipage}
    \end{center}
%    This example shows a Java HashMap being passed to a Python function, which retrieves and prints values.
\end{frame}

\begin{frame}[fragile]{Example 2: Java calling Python Script}

    \begin{minipage}[b]{0.45\linewidth}
        \centering
        \begin{lstlisting}[
            frame=single,
            language=python,
            numbers=left,
            basicstyle=\tiny\ttfamily,
            caption={\normalsize{Adding complex no. in loop}},
            captionpos=b,
            backgroundcolor=\color{lightgray},   % Background color
            numberstyle=\color{pink},            % Line number color
            keywordstyle=\color{keywordcolor}\bfseries, % Keywords color
            commentstyle=\color{commentcolor},   % Comment color
            stringstyle=\color{stringcolor}      % String color
        ,label={lst:lstlisting4}]
class ComplexNumber:
def __init__(self, r, i):
    self.r = r
    self.i = i
def __add__(self, other):
    cTmp= ComplexNumber(self.r, other.i)
    return ComplexNumber(self.r+other.r+cTmp.r,
                         self.i+other.i+cTmp.i)
def __str__(self):
    return f"({self.r} + {self.i}i)"

def addComplex(i:int, n:int):
    c1 = ComplexNumber(545,n)  #python object
    c2 = ComplexNumber(9878,i) #python object
    return c1 + c2

def addComplexLoop(n: int):
    sum=0
    for i in range(10000):
        sum= sum+ addComplex(i,n).r
    return sum
        \end{lstlisting}
    \end{minipage}
    \hspace{6mm}
    \begin{minipage}[b]{0.48\linewidth}
        \centering
        \begin{lstlisting}[
            frame=single,
            language=java,
            numbers=left,
            basicstyle=\tiny\ttfamily,
            caption={Python script called from java},
            captionpos=b,
            backgroundcolor=\color{lightgray},   % Background color
            numberstyle=\color{pink},            % Line number color
            keywordstyle=\color{keywordcolor}\bfseries, % Keywords color
            commentstyle=\color{commentcolor},   % Comment color
            stringstyle=\color{stringcolor}      % String color
        ,label={lst:lstlisting2.3}]
void test() {
    String pyFile=
       "./src/main/polyglot/testCase.py";
    //create a context
    try (Context ctx = Context.create()) {
       File f= new File(pyFile);
       //load the file
       Source src= Source.
                  newBuilder("python",f).build();
       //load the py script in ctx
       ctx.eval(src);
       //getting python function
       Value pyLoop = ctx.getBindings("python")
               .getMember("addComplexLoop");
       //calling python fun
       Value result= pyLoop.execute(1000);
       System.out.println(result.asLong());
    } catch (Exception e) {
       e.printStackTrace();
    }
}
        \end{lstlisting}
    \end{minipage}
\end{frame}
\begin{frame}[fragile]{Example 3: Java utilizing JavaScript's Library}
    \begin{center}
        \begin{minipage}{0.8\textwidth} % Adjust width as needed
            \begin{lstlisting}[
                frame=single,
                language=java,
                label={lst:jsInJava},
                numbers=left,
                basicstyle=\tiny\ttfamily,
                caption={Accessing the object returned by lodash in java},
                captionpos=b,
                backgroundcolor=\color{lightgray},   % Background color
                numberstyle=\color{pink},            % Line number color
                keywordstyle=\color{keywordcolor}\bfseries, % Keywords color
                commentstyle=\color{commentcolor},   % Comment color
                stringstyle=\color{stringcolor}      % String color
                ]
try (Context context = Context.newBuilder("js").allowAllAccess(true).build()) {
    // Load Lodash library from CDN
    String lodashLoad=
            "load('https://cdn.jsdelivr.net/npm/lodash@4.17.21/lodash.min.js');";
    context.eval("js", lodashLoad);
    // Define a JavaScript object and use Lodash to deep clone it
    String jsCode = ""
            const obj = {
                name: "Rosy",
                address: {
                    city: "Mumbai"
                }
            };
            _.cloneDeep(obj);
            "";

    Value clonedObject = context.eval("js", jsCode);
    // Access properties from the cloned object using the Value API
    String name = clonedObject.getMember("name").asString();
    String city = clonedObject.getMember("address").getMember("city").asString();
}
            \end{lstlisting}
        \end{minipage}
    \end{center}
\end{frame}

\begin{frame}{Why Polyglot Programming?}
    \begin{itemize}
        \item Enhances flexibility, choose languages based on their strengths.
        \vspace{3mm}
        \item Use of existing libraries, increases development speed.
        \vspace{3mm}
        \item \textbf{Use case}: Python for simplicity, Java for performance, and JavaScript for web integration.
    \end{itemize}
\end{frame}

\begin{frame}{Challenges with Polyglot Programs}
    \begin{itemize}
        \item \textbf{Polyglot runtimes} are complex, though they abstract language boundaries.
        \vspace{3mm}
        \item \textbf{Debugging \& Readability} of mixed-language programs can be challenging.
        \vspace{3mm}
        \item \textbf{Traditional analysis} tools may not fully support polyglot code analysis.
    \end{itemize}
\end{frame}

